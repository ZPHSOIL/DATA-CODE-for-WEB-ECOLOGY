\documentclass[]{article}
\usepackage{lmodern}
\usepackage{amssymb,amsmath}
\usepackage{ifxetex,ifluatex}
\usepackage{fixltx2e} % provides \textsubscript
\ifnum 0\ifxetex 1\fi\ifluatex 1\fi=0 % if pdftex
  \usepackage[T1]{fontenc}
  \usepackage[utf8]{inputenc}
\else % if luatex or xelatex
  \ifxetex
    \usepackage{mathspec}
  \else
    \usepackage{fontspec}
  \fi
  \defaultfontfeatures{Ligatures=TeX,Scale=MatchLowercase}
\fi
% use upquote if available, for straight quotes in verbatim environments
\IfFileExists{upquote.sty}{\usepackage{upquote}}{}
% use microtype if available
\IfFileExists{microtype.sty}{%
\usepackage{microtype}
\UseMicrotypeSet[protrusion]{basicmath} % disable protrusion for tt fonts
}{}
\usepackage[margin=1in]{geometry}
\usepackage{hyperref}
\hypersetup{unicode=true,
            pdftitle={Disperal gradient},
            pdfborder={0 0 0},
            breaklinks=true}
\urlstyle{same}  % don't use monospace font for urls
\usepackage{color}
\usepackage{fancyvrb}
\newcommand{\VerbBar}{|}
\newcommand{\VERB}{\Verb[commandchars=\\\{\}]}
\DefineVerbatimEnvironment{Highlighting}{Verbatim}{commandchars=\\\{\}}
% Add ',fontsize=\small' for more characters per line
\usepackage{framed}
\definecolor{shadecolor}{RGB}{248,248,248}
\newenvironment{Shaded}{\begin{snugshade}}{\end{snugshade}}
\newcommand{\KeywordTok}[1]{\textcolor[rgb]{0.13,0.29,0.53}{\textbf{#1}}}
\newcommand{\DataTypeTok}[1]{\textcolor[rgb]{0.13,0.29,0.53}{#1}}
\newcommand{\DecValTok}[1]{\textcolor[rgb]{0.00,0.00,0.81}{#1}}
\newcommand{\BaseNTok}[1]{\textcolor[rgb]{0.00,0.00,0.81}{#1}}
\newcommand{\FloatTok}[1]{\textcolor[rgb]{0.00,0.00,0.81}{#1}}
\newcommand{\ConstantTok}[1]{\textcolor[rgb]{0.00,0.00,0.00}{#1}}
\newcommand{\CharTok}[1]{\textcolor[rgb]{0.31,0.60,0.02}{#1}}
\newcommand{\SpecialCharTok}[1]{\textcolor[rgb]{0.00,0.00,0.00}{#1}}
\newcommand{\StringTok}[1]{\textcolor[rgb]{0.31,0.60,0.02}{#1}}
\newcommand{\VerbatimStringTok}[1]{\textcolor[rgb]{0.31,0.60,0.02}{#1}}
\newcommand{\SpecialStringTok}[1]{\textcolor[rgb]{0.31,0.60,0.02}{#1}}
\newcommand{\ImportTok}[1]{#1}
\newcommand{\CommentTok}[1]{\textcolor[rgb]{0.56,0.35,0.01}{\textit{#1}}}
\newcommand{\DocumentationTok}[1]{\textcolor[rgb]{0.56,0.35,0.01}{\textbf{\textit{#1}}}}
\newcommand{\AnnotationTok}[1]{\textcolor[rgb]{0.56,0.35,0.01}{\textbf{\textit{#1}}}}
\newcommand{\CommentVarTok}[1]{\textcolor[rgb]{0.56,0.35,0.01}{\textbf{\textit{#1}}}}
\newcommand{\OtherTok}[1]{\textcolor[rgb]{0.56,0.35,0.01}{#1}}
\newcommand{\FunctionTok}[1]{\textcolor[rgb]{0.00,0.00,0.00}{#1}}
\newcommand{\VariableTok}[1]{\textcolor[rgb]{0.00,0.00,0.00}{#1}}
\newcommand{\ControlFlowTok}[1]{\textcolor[rgb]{0.13,0.29,0.53}{\textbf{#1}}}
\newcommand{\OperatorTok}[1]{\textcolor[rgb]{0.81,0.36,0.00}{\textbf{#1}}}
\newcommand{\BuiltInTok}[1]{#1}
\newcommand{\ExtensionTok}[1]{#1}
\newcommand{\PreprocessorTok}[1]{\textcolor[rgb]{0.56,0.35,0.01}{\textit{#1}}}
\newcommand{\AttributeTok}[1]{\textcolor[rgb]{0.77,0.63,0.00}{#1}}
\newcommand{\RegionMarkerTok}[1]{#1}
\newcommand{\InformationTok}[1]{\textcolor[rgb]{0.56,0.35,0.01}{\textbf{\textit{#1}}}}
\newcommand{\WarningTok}[1]{\textcolor[rgb]{0.56,0.35,0.01}{\textbf{\textit{#1}}}}
\newcommand{\AlertTok}[1]{\textcolor[rgb]{0.94,0.16,0.16}{#1}}
\newcommand{\ErrorTok}[1]{\textcolor[rgb]{0.64,0.00,0.00}{\textbf{#1}}}
\newcommand{\NormalTok}[1]{#1}
\usepackage{graphicx,grffile}
\makeatletter
\def\maxwidth{\ifdim\Gin@nat@width>\linewidth\linewidth\else\Gin@nat@width\fi}
\def\maxheight{\ifdim\Gin@nat@height>\textheight\textheight\else\Gin@nat@height\fi}
\makeatother
% Scale images if necessary, so that they will not overflow the page
% margins by default, and it is still possible to overwrite the defaults
% using explicit options in \includegraphics[width, height, ...]{}
\setkeys{Gin}{width=\maxwidth,height=\maxheight,keepaspectratio}
\IfFileExists{parskip.sty}{%
\usepackage{parskip}
}{% else
\setlength{\parindent}{0pt}
\setlength{\parskip}{6pt plus 2pt minus 1pt}
}
\setlength{\emergencystretch}{3em}  % prevent overfull lines
\providecommand{\tightlist}{%
  \setlength{\itemsep}{0pt}\setlength{\parskip}{0pt}}
\setcounter{secnumdepth}{0}
% Redefines (sub)paragraphs to behave more like sections
\ifx\paragraph\undefined\else
\let\oldparagraph\paragraph
\renewcommand{\paragraph}[1]{\oldparagraph{#1}\mbox{}}
\fi
\ifx\subparagraph\undefined\else
\let\oldsubparagraph\subparagraph
\renewcommand{\subparagraph}[1]{\oldsubparagraph{#1}\mbox{}}
\fi

%%% Use protect on footnotes to avoid problems with footnotes in titles
\let\rmarkdownfootnote\footnote%
\def\footnote{\protect\rmarkdownfootnote}

%%% Change title format to be more compact
\usepackage{titling}

% Create subtitle command for use in maketitle
\newcommand{\subtitle}[1]{
  \posttitle{
    \begin{center}\large#1\end{center}
    }
}

\setlength{\droptitle}{-2em}

  \title{Disperal gradient}
    \pretitle{\vspace{\droptitle}\centering\huge}
  \posttitle{\par}
    \author{}
    \preauthor{}\postauthor{}
    \date{}
    \predate{}\postdate{}
  

\begin{document}
\maketitle

\begin{Shaded}
\begin{Highlighting}[]
\NormalTok{SPNETDISP <-}\StringTok{ }\KeywordTok{read_excel}\NormalTok{(}\StringTok{"Desktop/SPATIAL NETWORK /SPATIONETWORK/SPNETDISP.xlsx"}\NormalTok{)}
\CommentTok{#subset the data}
\NormalTok{SPNETDISP_BF<-}\KeywordTok{subset}\NormalTok{(SPNETDISP,FncG}\OperatorTok{==}\StringTok{"BF"}\NormalTok{) }\CommentTok{#Subset BF}
\NormalTok{SPNETDISP_G1<-}\KeywordTok{subset}\NormalTok{(SPNETDISP,Grid}\OperatorTok{==}\StringTok{"R1.29"}\NormalTok{)}\CommentTok{#SUBSET GRID 1.29}
\end{Highlighting}
\end{Shaded}

\section{plot per functional group
grid1}\label{plot-per-functional-group-grid1}

\includegraphics{Dispersal-funcg_files/figure-latex/unnamed-chunk-3-1.pdf}

\section{plot per functional group
grid2}\label{plot-per-functional-group-grid2}

\includegraphics{Dispersal-funcg_files/figure-latex/unnamed-chunk-5-1.pdf}

\section{plot per functional group
grid3}\label{plot-per-functional-group-grid3}

\includegraphics{Dispersal-funcg_files/figure-latex/unnamed-chunk-6-1.pdf}

\section{plot per functional group
grid4}\label{plot-per-functional-group-grid4}

\includegraphics{Dispersal-funcg_files/figure-latex/unnamed-chunk-7-1.pdf}

\section{plot per functional group
grid5}\label{plot-per-functional-group-grid5}

\includegraphics{Dispersal-funcg_files/figure-latex/unnamed-chunk-8-1.pdf}

\section{plot per functional group
grid6}\label{plot-per-functional-group-grid6}

\includegraphics{Dispersal-funcg_files/figure-latex/unnamed-chunk-9-1.pdf}

\section{plot per functional group
grid7}\label{plot-per-functional-group-grid7}

\includegraphics{Dispersal-funcg_files/figure-latex/unnamed-chunk-10-1.pdf}

\section{plot per functional group
grid8}\label{plot-per-functional-group-grid8}

\includegraphics{Dispersal-funcg_files/figure-latex/unnamed-chunk-11-1.pdf}

\section{plot per functional group
grid9}\label{plot-per-functional-group-grid9}

\includegraphics{Dispersal-funcg_files/figure-latex/unnamed-chunk-12-1.pdf}

\section{plot per functional group
grid10}\label{plot-per-functional-group-grid10}

\includegraphics{Dispersal-funcg_files/figure-latex/unnamed-chunk-13-1.pdf}

\section{plot per Grid BF}\label{plot-per-grid-bf}

\includegraphics{Dispersal-funcg_files/figure-latex/unnamed-chunk-14-1.pdf}

\section{plot per Grid ROTIFER}\label{plot-per-grid-rotifer}

\includegraphics{Dispersal-funcg_files/figure-latex/unnamed-chunk-15-1.pdf}

\section{plot per Grid FF}\label{plot-per-grid-ff}

\includegraphics{Dispersal-funcg_files/figure-latex/unnamed-chunk-16-1.pdf}

\section{plot per Grid PF}\label{plot-per-grid-pf}

\includegraphics{Dispersal-funcg_files/figure-latex/unnamed-chunk-17-1.pdf}

\section{plot per Grid predator}\label{plot-per-grid-predator}

\begin{Shaded}
\begin{Highlighting}[]
  \KeywordTok{ggplot}\NormalTok{(}\DataTypeTok{data=}\NormalTok{ SPNETDISP_PRE, }\DataTypeTok{mapping =} \KeywordTok{aes}\NormalTok{(}\DataTypeTok{x=}\NormalTok{ Time, }\DataTypeTok{y=}\NormalTok{ Disp, }\DataTypeTok{color =}\NormalTok{ Grid, }\DataTypeTok{group=}\NormalTok{ Grid)) }\OperatorTok{+}
\StringTok{ }\KeywordTok{geom_point}\NormalTok{()}\OperatorTok{+}\KeywordTok{geom_line}\NormalTok{()}\OperatorTok{+}
\StringTok{   }\KeywordTok{labs}\NormalTok{(}\DataTypeTok{title =}\StringTok{"Predator dispersal change between grids "}\NormalTok{, }\DataTypeTok{x =}\StringTok{"Time points"}\NormalTok{)}\OperatorTok{+}
\KeywordTok{theme_classic}\NormalTok{()}
\end{Highlighting}
\end{Shaded}

\includegraphics{Dispersal-funcg_files/figure-latex/unnamed-chunk-18-1.pdf}


\end{document}
